% Options for packages loaded elsewhere
\PassOptionsToPackage{unicode}{hyperref}
\PassOptionsToPackage{hyphens}{url}
\PassOptionsToPackage{dvipsnames,svgnames,x11names}{xcolor}
%
\documentclass[
  letterpaper,
  paper=6in:9in,pagesize=pdftex,footinclude=on,11pt]{scrreprt}

\usepackage{amsmath,amssymb}
\usepackage{iftex}
\ifPDFTeX
  \usepackage[T1]{fontenc}
  \usepackage[utf8]{inputenc}
  \usepackage{textcomp} % provide euro and other symbols
\else % if luatex or xetex
  \usepackage{unicode-math}
  \defaultfontfeatures{Scale=MatchLowercase}
  \defaultfontfeatures[\rmfamily]{Ligatures=TeX,Scale=1}
\fi
\usepackage[]{libertinus}
\ifPDFTeX\else  
    % xetex/luatex font selection
    \setmonofont[Scale=0.7]{Consolas}
\fi
% Use upquote if available, for straight quotes in verbatim environments
\IfFileExists{upquote.sty}{\usepackage{upquote}}{}
\IfFileExists{microtype.sty}{% use microtype if available
  \usepackage[]{microtype}
  \UseMicrotypeSet[protrusion]{basicmath} % disable protrusion for tt fonts
}{}
\usepackage{xcolor}
\setlength{\emergencystretch}{3em} % prevent overfull lines
\setcounter{secnumdepth}{5}
% Make \paragraph and \subparagraph free-standing
\makeatletter
\ifx\paragraph\undefined\else
  \let\oldparagraph\paragraph
  \renewcommand{\paragraph}{
    \@ifstar
      \xxxParagraphStar
      \xxxParagraphNoStar
  }
  \newcommand{\xxxParagraphStar}[1]{\oldparagraph*{#1}\mbox{}}
  \newcommand{\xxxParagraphNoStar}[1]{\oldparagraph{#1}\mbox{}}
\fi
\ifx\subparagraph\undefined\else
  \let\oldsubparagraph\subparagraph
  \renewcommand{\subparagraph}{
    \@ifstar
      \xxxSubParagraphStar
      \xxxSubParagraphNoStar
  }
  \newcommand{\xxxSubParagraphStar}[1]{\oldsubparagraph*{#1}\mbox{}}
  \newcommand{\xxxSubParagraphNoStar}[1]{\oldsubparagraph{#1}\mbox{}}
\fi
\makeatother


\providecommand{\tightlist}{%
  \setlength{\itemsep}{0pt}\setlength{\parskip}{0pt}}\usepackage{longtable,booktabs,array}
\usepackage{calc} % for calculating minipage widths
% Correct order of tables after \paragraph or \subparagraph
\usepackage{etoolbox}
\makeatletter
\patchcmd\longtable{\par}{\if@noskipsec\mbox{}\fi\par}{}{}
\makeatother
% Allow footnotes in longtable head/foot
\IfFileExists{footnotehyper.sty}{\usepackage{footnotehyper}}{\usepackage{footnote}}
\makesavenoteenv{longtable}
\usepackage{graphicx}
\makeatletter
\newsavebox\pandoc@box
\newcommand*\pandocbounded[1]{% scales image to fit in text height/width
  \sbox\pandoc@box{#1}%
  \Gscale@div\@tempa{\textheight}{\dimexpr\ht\pandoc@box+\dp\pandoc@box\relax}%
  \Gscale@div\@tempb{\linewidth}{\wd\pandoc@box}%
  \ifdim\@tempb\p@<\@tempa\p@\let\@tempa\@tempb\fi% select the smaller of both
  \ifdim\@tempa\p@<\p@\scalebox{\@tempa}{\usebox\pandoc@box}%
  \else\usebox{\pandoc@box}%
  \fi%
}
% Set default figure placement to htbp
\def\fps@figure{htbp}
\makeatother
% definitions for citeproc citations
\NewDocumentCommand\citeproctext{}{}
\NewDocumentCommand\citeproc{mm}{%
  \begingroup\def\citeproctext{#2}\cite{#1}\endgroup}
\makeatletter
 % allow citations to break across lines
 \let\@cite@ofmt\@firstofone
 % avoid brackets around text for \cite:
 \def\@biblabel#1{}
 \def\@cite#1#2{{#1\if@tempswa , #2\fi}}
\makeatother
\newlength{\cslhangindent}
\setlength{\cslhangindent}{1.5em}
\newlength{\csllabelwidth}
\setlength{\csllabelwidth}{3em}
\newenvironment{CSLReferences}[2] % #1 hanging-indent, #2 entry-spacing
 {\begin{list}{}{%
  \setlength{\itemindent}{0pt}
  \setlength{\leftmargin}{0pt}
  \setlength{\parsep}{0pt}
  % turn on hanging indent if param 1 is 1
  \ifodd #1
   \setlength{\leftmargin}{\cslhangindent}
   \setlength{\itemindent}{-1\cslhangindent}
  \fi
  % set entry spacing
  \setlength{\itemsep}{#2\baselineskip}}}
 {\end{list}}
\usepackage{calc}
\newcommand{\CSLBlock}[1]{\hfill\break\parbox[t]{\linewidth}{\strut\ignorespaces#1\strut}}
\newcommand{\CSLLeftMargin}[1]{\parbox[t]{\csllabelwidth}{\strut#1\strut}}
\newcommand{\CSLRightInline}[1]{\parbox[t]{\linewidth - \csllabelwidth}{\strut#1\strut}}
\newcommand{\CSLIndent}[1]{\hspace{\cslhangindent}#1}

\usepackage{geometry}
\usepackage{wrapfig}
\usepackage{fvextra}
\DefineVerbatimEnvironment{Highlighting}{Verbatim}{breaklines,commandchars=\\\{\}}
\geometry{
    paperwidth=6in,
    paperheight=9in,
    textwidth=4.5in, % Adjust this to your preferred text width
    textheight=6.5in,  % Adjust this to your preferred text height
    inner=0.75in,    % Adjust margins as needed
    outer=0.75in,
    top=0.75in,
    bottom=1in
}
\usepackage{makeidx}
\usepackage{tabularx}
\usepackage{float}
\usepackage{graphicx}
\usepackage{array}
\graphicspath{{diagrams/}}
\makeindex
\makeatletter
\@ifpackageloaded{bookmark}{}{\usepackage{bookmark}}
\makeatother
\makeatletter
\@ifpackageloaded{caption}{}{\usepackage{caption}}
\AtBeginDocument{%
\ifdefined\contentsname
  \renewcommand*\contentsname{Table of contents}
\else
  \newcommand\contentsname{Table of contents}
\fi
\ifdefined\listfigurename
  \renewcommand*\listfigurename{List of Figures}
\else
  \newcommand\listfigurename{List of Figures}
\fi
\ifdefined\listtablename
  \renewcommand*\listtablename{List of Tables}
\else
  \newcommand\listtablename{List of Tables}
\fi
\ifdefined\figurename
  \renewcommand*\figurename{Figure}
\else
  \newcommand\figurename{Figure}
\fi
\ifdefined\tablename
  \renewcommand*\tablename{Table}
\else
  \newcommand\tablename{Table}
\fi
}
\@ifpackageloaded{float}{}{\usepackage{float}}
\floatstyle{ruled}
\@ifundefined{c@chapter}{\newfloat{codelisting}{h}{lop}}{\newfloat{codelisting}{h}{lop}[chapter]}
\floatname{codelisting}{Listing}
\newcommand*\listoflistings{\listof{codelisting}{List of Listings}}
\makeatother
\makeatletter
\makeatother
\makeatletter
\@ifpackageloaded{caption}{}{\usepackage{caption}}
\@ifpackageloaded{subcaption}{}{\usepackage{subcaption}}
\makeatother
\makeatletter
\@ifpackageloaded{tcolorbox}{}{\usepackage[skins,breakable]{tcolorbox}}
\makeatother
\makeatletter
\@ifundefined{shadecolor}{\definecolor{shadecolor}{HTML}{000000}}{}
\makeatother
\makeatletter
\@ifundefined{codebgcolor}{\definecolor{codebgcolor}{HTML}{f0f0f0}}{}
\makeatother
\makeatletter
\ifdefined\Shaded\renewenvironment{Shaded}{\begin{tcolorbox}[enhanced, boxrule=0pt, colback={codebgcolor}, frame hidden, borderline west={3pt}{0pt}{shadecolor}, breakable, sharp corners]}{\end{tcolorbox}}\fi
\makeatother

\usepackage{bookmark}

\IfFileExists{xurl.sty}{\usepackage{xurl}}{} % add URL line breaks if available
\urlstyle{same} % disable monospaced font for URLs
\hypersetup{
  pdftitle={bayes-theorem-book},
  pdfauthor={Muthukrishnan},
  colorlinks=true,
  linkcolor={black},
  filecolor={Maroon},
  citecolor={Blue},
  urlcolor={blue},
  pdfcreator={LaTeX via pandoc}}


\title{bayes-theorem-book}
\author{Muthukrishnan}
\date{2025-09-02}

\begin{document}

\maketitle

\newpage

%----------------------------------------------
%   Copyright
%----------------------------------------------

\begin{flushleft}
\Large Copyright
\end{flushleft}

\vspace*{\baselineskip}

\begin{flushleft}
\large Cover Image
\large Copyright 2023 Muthukrishnan. All Rights Reserved 
\end{flushleft}



\vspace{5mm} %5mm vertical space

The triangle featured on the cover page is known as the Penrose Triangle, also referred to as the impossible triangle. It is an optical illusion that depicts a three-dimensional figure in the form of a triangular loop, which, despite its appearance, cannot be constructed in a physical three-dimensional space. It was created by a Swedish artist Oscar Reutersvärd in the 1930s and was later brought to prominence by mathematician Roger Penrose in the 1950s. Although the Penrose Triangle cannot be realized as a physical object, it represents an intriguing example of how visual perception can be deceived to see an ostensibly impossible form. It has emerged as an iconic figure in the realm of optical illusions and is frequently employed to demonstrate ideas related to paradoxes and the visual perception in the fields of art and design.


\vspace{5mm} %5mm vertical space

\begin{flushleft}
\large Essential Search Algorithms: Navigating the Digital Maze
\large Copyright 2023 Muthukrishnan. All Rights Reserved
\end{flushleft}

\vspace{5mm} %5mm vertical space

No part of this publication may be reproduced or transmitted in any form whatsoever, electronic, or mechanical, including photocopying, recording, or by any informational storage or retrieval system without express written, dated and signed permission from the author.

\newpage

%----------------------------------------------
%   Preface
%----------------------------------------------

\begin{flushleft}
\Large Preface
\end{flushleft}

\vspace*{\baselineskip}
Ever wonder how Google finds your perfect search result in milliseconds? Or how your GPS maps the fastest route to your destination? It's all thanks to the clever search algorithms invented by great software engineers who needed to solve some complex search problems.
\\
\\
Search algorithms are everywhere! From finding your friend's profile on Facebook to suggesting the best route on a map, these algorithms are constantly sifting through mountains of data, searching for what you're looking for. Search is at the core of all these real-world technology problems.
\\
\\
This book is your guide to essential search methods used across different fields. I've carefully curated these algorithms from a vast collection, selecting those that form the building blocks. Understanding and applying these algorithms will provide you with the necessary baseline to explore more complex ones. Plus, you might even be able to create your own algorithm tailored to the problem you're solving. Each section explains the ideas clearly and includes Python code you can use in your projects. All the code is free and available on my GitHub page under the MIT license.
\\
\\
By the end, you'll have the skills to explore both simple and complex decision-making. Whether it's managing supply chains or designing smart robots, these essential search methods will guide you through tough challenges and help you discover solutions you didn't know were there.
\\
\\
So, join me on a journey to understand how search works. Let's get started!
\\
\\
Muthukrishnan\\
Bangalore, India\\
2023
\newpage

%----------------------------------------------
%   Dedication
%----------------------------------------------


\vspace*{\baselineskip}
\vspace{20mm} %5mm vertical space
\textit{For my parents, Amrita and Atharva}

\newpage
\renewcommand*\contentsname{Table of contents}
{
\hypersetup{linkcolor=}
\setcounter{tocdepth}{1}
\tableofcontents
}

\bookmarksetup{startatroot}

\chapter*{Preface}\label{preface}
\addcontentsline{toc}{chapter}{Preface}

\markboth{Preface}{Preface}

This is a Quarto book.

To learn more about Quarto books visit
\url{https://quarto.org/docs/books}.

\bookmarksetup{startatroot}

\chapter{Introduction}\label{introduction}

This is a book created from markdown and executable code.

See Knuth (1984) for additional discussion of literate programming.

\part{Introduction to Bayesian Thinking}

\chapter{Understanding Probability}\label{understanding-probability}

\chapter{Bayes' Theorem Fundamentals}\label{bayes-theorem-fundamentals}

\chapter{Setting Up Python
Environment}\label{setting-up-python-environment}

\part{Mathematical Foundations}

\chapter{Probability Theory
Essentials}\label{probability-theory-essentials}

\chapter{Statistical Concepts}\label{statistical-concepts}

\chapter{Linear Algebra Review}\label{linear-algebra-review}

\part{Implementing Bayes' Theorem}

\chapter{Basic Implementation}\label{basic-implementation}

\chapter{Working with Continuous
Distributions}\label{working-with-continuous-distributions}

\chapter{Discrete Probability
Examples}\label{discrete-probability-examples}

\part{Bayesian Inference}

\chapter{Parameter Estimation}\label{parameter-estimation}

\chapter{Conjugate Priors}\label{conjugate-priors}

\chapter{Prior Selection}\label{prior-selection}

\part{Markov Chain Monte Carlo (MCMC)}

\chapter{Introduction to MCMC}\label{introduction-to-mcmc}

\chapter{MCMC Algorithms}\label{mcmc-algorithms}

\chapter{Implementation with PyMC3}\label{implementation-with-pymc3}

\part{Practical Applications}

\chapter{A/B Testing}\label{ab-testing}

\chapter{Text Classification}\label{text-classification}

\chapter{Medical Diagnosis}\label{medical-diagnosis}

\part{Advanced Topics}

\chapter{Hierarchical Bayesian
Models}\label{hierarchical-bayesian-models}

\chapter{Bayesian Neural Networks}\label{bayesian-neural-networks}

\chapter{Gaussian Processes}\label{gaussian-processes}

\part{Real-World Applications}

\chapter{Finance}\label{finance}

\chapter{Marketing}\label{marketing}

\chapter{Scientific Applications}\label{scientific-applications}

\part{Best Practices and Advanced Tools}

\chapter{Code Organization}\label{code-organization}

\chapter{Performance Optimization}\label{performance-optimization}

\chapter{Modern Bayesian Libraries}\label{modern-bayesian-libraries}

\bookmarksetup{startatroot}

\chapter{Summary}\label{summary}

In summary, this book has no content whatsoever.

\bookmarksetup{startatroot}

\chapter*{References}\label{references}
\addcontentsline{toc}{chapter}{References}

\markboth{References}{References}

\phantomsection\label{refs}
\begin{CSLReferences}{1}{0}
\bibitem[\citeproctext]{ref-knuth84}
Knuth, Donald E. 1984. {``Literate Programming.''} \emph{Comput. J.} 27
(2): 97--111. \url{https://doi.org/10.1093/comjnl/27.2.97}.

\end{CSLReferences}



\printindex


\end{document}
